Последний вопрос --- как сэмплировать термальные скорости.

Будем считать, что форм фактор $f(q,v)$ медленно меняется и его можно сэмплировать мажорантой $f(q,v) = const$.

Остается только часть с $\vec{v}_1$. В вероятность входит величина 

$$
|\vec{v}_1' - \vec{V}'| 2\pi \cfrac{v_1^2}{(2\pi)^{3/2} v_T^3} dv_1 e^{-v_1^2/2v_T^2} d\cos{\theta} = \sqrt{ v_1^2 + V^2 - 2 v_1 V \cos{\theta} \pm \Delta^2 } \frac{ v_1^2}{v_T^3 \sqrt{2\pi}} dv_1 e^{-v_1^2/2v_T^2} d\cos{\theta}
$$

Существует несколько случаев
\begin{itemize}
	\item $+\Delta^2$ либо $-\Delta^2$, но $v_1 > V+\Delta$ либо $0 < v_1 < V-\Delta$
	
	Тогда $\cos{\theta}$ меняется от $-1$ до $1$. 
	
	\item $-\Delta^2$ и $|\Delta - V| < v_1 < V+\Delta$
	
	Тогда  $\cos{\theta}$ меняется от $-1$ до $\cos{\theta}^{max}$
\end{itemize}

В первом случае берется интеграл по $\cos{\theta}$. Он имеет вид
$$
	\sqrt{A - B y} dy = dG(y) 	
$$
где 
$$
	G(y) = 2\frac{(A+B)^{3/2} - (A-By)^{3/2}}{3B}
$$
Остюда легко сэмлировать $y$, обращая $G(y)/G(1)$

$$
\xi_{\theta} = \frac{G(y)}{G(1)} = \cfrac{(1+t)^{3/2} - (1-t y)^{3/2}} {(1+t)^{3/2} - (1-t )^{3/2}}, t = \frac{B}{A} < 1
$$

$$
	y = \cfrac{1 - ( (1+t)^{3/2} - \xi \left[ (1+t)^{3/2}-(1-t)^{3/2}\right] )^{2/3} }{t}
$$
Далее наш интеграл свелся к 

$$
G(1) \frac{ v_1^2}{v_T^3\sqrt{2\pi}} dv_1 e^{-v_1^2/2v_T^2} d\xi_{\theta}
$$

$$G(1) =2 \sqrt{A} \cdot \left[ Q(t=B/A) = \cfrac{(1 + t)^{3/2} - (1 - t)^{3/2}}{3t} \right]$$

$$
 = G(1) \sqrt{\cfrac{u}{\pi}} e^{-u}du
$$

Теперь рассмотрим эндотермический случай, когда $v \in [|V-\Delta|, V+\Delta]$.
тогда $\cos{\theta}$ ограничен $\cos{\theta}^{max}$
$$
	\cos{\theta}^{max} = \cfrac{V^2+v_1^2-\Delta^2}{2 V v_1} = A/B = t
$$

тогда вместо $G(1)$ будет стоять $G(y^{max}) = 2 (A+B)^{3/2}/3B = 2/3\sqrt{B} (t+1)^{3/2}$ 

$$
\xi_{\theta} = \frac{G(y)}{G(y^{max})} = \cfrac{(t+1)^{3/2} - (t - y)^{3/2}} {(1+t)^{3/2}}
$$

$$
	y = t - \left( (1+t)^{3/2} (1-\xi) \right)^{2/3}
$$

\subsubsection{Детали реализации}

Мы возьмём сетку по $e,l$ в виде квадродерева.
В узлах сетки будут предварительно рассчитанные массивы $P(e_m,l_m)[i]$, равные 
полной вероятности столконовения частицы на траектории с ядром типа $i$.

Вероятность столкновения в точке $(e,l)$ будет аппроксимироваться логарифмически

\[	
	\ln p(e,l) \approx \sum_{m=0}^{4}{a_m \ln p(e_m,l_m)}
\]

Далее определяется случайное время $\tau = -\ln \xi / {p_{a, total}(e,l)}$ ($\xi$ распределен равномерно)

Если $\tau$ не превышает оставшееся время симуляции $T$, то мы разыгрываем столкновение
и присваиваем $T = T - \tau$

Потом разыгрывается элемент с вероятностью пропорциональной $p(e,l)[i]$

После этого нужно определить место столкновения. Для этого есть еще одна сетка в координатах
$(r,v)$ для каждого элемента, где находится информация о вероятности столкновения за единицу времени на элементе $i$.

\[
	p[i](r,v) = \frac{dP}{d\tau}(r,v)
\]

(величина в точке $r,v$ интерполируется так же: логарифмически )

После этого строится распределение по параметру $\theta$, а именно

\[
	\frac{dP}{d\theta} = \frac{dP}{d\tau} \frac{d\tau}{d\theta}
\]

и выбирается  $\theta$ с соответствующей вероятностью.

Теперь, когда есть $\theta$, а следовательно и $(r,v)$, идет столкновение с ядром $i$ в этой точке.

Для столкновения мы разыгрываем скорость ядра, но делаем это с вероятностью 

\[
	p \sim e^{-\frac{v^2}{2mT}}
\]

Таким образом у нас еще остается фактор от меры $p_m$. Затем мы генерируем остальные параметры столкновения (углы $\theta_{in}$, $\theta_{out}$, $\phi_{out}$) и вычисляем форм фактор от столкновения $f(q)$.

В итоге у нас появился неучтенный фактор 

\[
	F = p_m\cdot f(q)
\]

из-за этого фактора у нас неправильно сэмплируются столкновения, поэтому мы запоминаем для $F$ мажоранту $M_F(r,v)$ (на этапе расчета $p(r,v)$) и принимаем столкновение с вероятностью $F/M_F(r,v)$, а в противном случае, разыгрываем столкновение в точке $r,v$ заново, пока мы его не примем.

(для увеличения эффективности интервал скоростей $v_1$ ядер разбивается на части и в каждой части мажоранта находится отдельно).

Когда мы создаем решетку у нас такие параметры:

1) $L_0$ --- начальный уровень (число бинов $2^{L_0} \times 2^{L_0}$)
2) $p_{0}$ --- порогавая вероятность
3) $\delta_0 = 2$ --- разница вероятностей, считающаяся большой.
4) $\delta_{acc}$ --- разница вероятностей, считающаяся приемлемой.
5) $N_{max}$ --- максимальное число бинов

Алгоритм такой:

\begin{itemize}
	\item сначала сетка имеет размер $2^{L_0} \times 2^{L_0}$
	\item затем, пока $N<N_{max}/2$ или уточнять не нужно мы уточняем бины, если относительная разница величин на его краях больше $\delta_{acc}$
	\item затем пока $N<N_{max}/2$ или уточнять не нужно  мы уточняем бины, если относительная разница величин на его краях больше $\delta_{0}$ (если все значения менее порога, то не уточняем)
	\item Если еще $N<N_{max}/2$, то утоочняем по $\delta_{acc}$
\end{itemize}

\subsubsection{Самодействие}

Итак, дискретное уравнение выглядит следующим образом:

\begin{equation}
	\frac{dN_i}{dt} = C_i + S_{ij} N_j + A_{ijk} N_j N_k
\end{equation}

Предлагается решать это методом среднего поля: при малом шаге во времени $dt$ часть с 
$A_{ijk} N_j N_k$ выглядит так:

\[
	A_{ijk} N_j N_k = [ A_{ijk} N_k(t_0) ] N_j(t) + O(dt)
\]

т.е. мы рассеиваим наши частицы на неизменяющейся мишени $[ A_{ijk} N_k(t_0) ]$

Для того, чтобы найти вероятность рассеяния $p(r,v)$ мы наши замороженны частицы каждый раз должны строить гитограмму в плоскости $r,v,\cos(\theta)$ где $\cos(\theta) = v_r/v$

Далее мы можем найти вероятность столкновения:

\[
	\frac{dP}{dt} = \int{dcos(\theta) dv d\varphi \; n(r,v,\cos(\theta)) P(|\vec{v}-\vec{v}'|) }
\]

где
 
\[
	P(|\vec{v}-\vec{v}'|) = \sigma |\vec{v}-\vec{v}'| = \sigma_0 \frac{\mathcal{M}^2}{\mathcal{M}_0^2} 
	\sqrt{v^2 + v'^2-2 v_r v_r' \eta -2 v_{tau} v_{tau}' \cos \varphi - \frac{4\delta}{M}}
\]

мы считаем, что распределение равномерное по $\varphi$ (есть изотропность)

Обратить внимание нужно вот на что: если мы возьмем 1 траекторию, то она создает концентрацию

\[
	n(r) = \frac{1}{3r^2} \frac{1}{v_r(\tau)}
\]

--- очень плохо ведущую себя функцию. Предполагается, что бинирование по $r$ загладит все.

